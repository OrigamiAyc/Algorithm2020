\documentclass[]{report}
\usepackage[hmargin=1.25in,vmargin=1in]{geometry} %调整页边距
% \usepackage[inner=1in,outer=1.25in]{geometry} %书籍左右不等宽排版
\usepackage[utf8]{inputenc}
\usepackage[]{ctex} %据说可以直接调用诸如 \kaishu \fangsong \heiti 的命令修改字体
\usepackage[svgnames]{xcolor} % Using colors
% \usepackage{background} % To include background images
\usepackage{fancyhdr} % Needed to define custom headers/footers
\usepackage[]{xeCJK}
\setCJKmainfont[BoldFont = STHeiti, ItalicFont = STKaiti]{Songti SC Light} %中文主字体
\setCJKsansfont[BoldFont = Weibei SC, ItalicFont = HanziPen SC]{Xingkai SC Light} %中文无衬线字体
\setCJKmonofont[BoldFont = Libian SC, ItalicFont = STFangsong]{Yuanti SC Light} %中文等宽字体
\setmainfont{Times New Roman} %\rmfamily
\setsansfont[ItalicFont = American Typewriter]{Comic Sans MS} %\sffamily
\setmonofont{Courier} %\ttfamily
\newfontfamily\monaco{Courier} % 用于代码段字体设置
\newfontfamily\OldCaption{Bodoni 72 Smallcaps Book} %用于全大写字母的标题
\usepackage{titlesec}
% 一下为适用于笔记/整理的板式(第几章-第几节)
% \titleformat{\chapter}{\centering\huge\bfseries}{第~\thechapter~章}{1em}{}
% \titleformat{\section}{\Large\bfseries}{第~\thesection~节}{1em}{}
% 一下为适用于作业的板式(第几次-第几题-abcd问)
\titleformat{\chapter}{\centering\huge\bfseries}{第六次作业}{1em}{}
\titleformat{\section}{\Large\bfseries}{第~\thesection~题}{1em}{}
\renewcommand{\thesubsection}{(\alph{subsection})}
\usepackage{lipsum} %填充文本

\usepackage{ulem} %解决下划线、删除线之类的
\usepackage{listings}
\lstset{
language=C++,
numberstyle = \monaco,
basicstyle = \monaco,
keywordstyle = \color{blue}\bfseries,
commentstyle=\color[HTML]{006400},
tabsize = 4,
%backgroundcolor=\color{bg}
emph = {int,float,double,char},emphstyle=\color{cyan},
emph = {[2]const, typedef},emphstyle = {[2]\color{red}} }

\makeatletter
\newif\if@restonecol
\makeatother
\let\algorithm\relax
\let\endalgorithm\relax
\usepackage[linesnumbered,ruled,vlined]{algorithm2e}%[ruled,vlined]{
\usepackage{algpseudocode}
\usepackage{amsmath}
\renewcommand{\algorithmicrequire}{\textbf{Input:}}  % Use Input in the format of Algorithm
\renewcommand{\algorithmicensure}{\textbf{Output:}} % Use Output in the format of Algorithm

\usepackage{amsmath} %数学公式问题
\usepackage{amsthm} %公式环境,如proof
\usepackage{booktabs} %三线表
\newcommand{\tabincell}[2]{\begin{tabular}{@{}#1@{}}#2\end{tabular}} %解决单元格内部换行的问题
% 比如这个 Beijing & 0,5 & 1,6 & 2,7 & 3,8 & 4,9 & The number changes every 3 months \\
% 改成这个 \tabincell{l}{Beijing}& \tabincell{c}{0,5}& \tabincell{c}{1,6}& \tabincell{c}{2,7}& \tabincell{c}{3,8}& \tabincell{c}{4,9}& \tabincell{c}{The number changes \\ every 3 months} \\
% 一个单元格过长,整行都需要修改
% 可以配合 \resizebox*{h-width}{v-width}{contents, e.g.tabular} 使用

\usepackage{mathrsfs} %在公式里面使用那个最花的字体
\usepackage{amssymb} %公式里面用空心黑体和旧式字体
\usepackage{amssymb} %AMS符号
\usepackage{amsthm} %AMS定理环境

\usepackage{markdown} %使用markdown语法,在编译时需要打开 shell-escape 标记,即 $ xelatex --shell-escape example.tex
\markdownSetup{hashEnumerators = true} %允许使用 #. 的方式编写有序列表
\markdownSetup{inlineFootnotes = true} %允许使用脚注形式的超链接,调用语法为 [anchor](uri), ^[footnote], <uri>
\markdownSetup{fencedCode = true} %以反引号和缩进来插入代码段,相当于 verbatim
\markdownSetup{
  pipeTables = true
} %支持表格的用法 (图片已经在markdown包里面支持了)
% \usepackage{booktabs} %解决三线表的线条粗细问题

\usepackage{graphicx} %插入图片
\usepackage{pdfpages} %插入PDF文件
\usepackage{makeidx}

\usepackage{tikz} %带圈字符
\usepackage{etoolbox} %带圈字符 (提供robustify)
\usepackage{enumitem}
\newcommand*{\circled}[1]{\lower.7ex\hbox{\tikz\draw (0pt, 0pt)%
    circle (.5em) node {\makebox[1em][c]{\small #1}};}} %新定义命令:带圈字符
\robustify{\circled}
% \usepackage{enumerate} %有序列表

\usepackage{hyperref} %超链接
% \usepackage[hidelinks]{hyperref} %隐藏超链接的红框
\markdownSetup{
  inlineFootnotes = true,
  renderers = {
    link = {\href{#3}{#1}},
  }
} % markdown块中使用直接点进去的超链接
% \setlist[enumerate,1]{label=(\arabic*).,font=\textup,leftmargin=7mm,labelsep=1.5mm,topsep=0mm,itemsep=-0.8mm}
% \setlist[enumerate,2]{label=(\alph*).,font=\textup,leftmargin=7mm,labelsep=1.5mm,topsep=-0.8mm,itemsep=-0.8mm}

\usepackage{braket}

%%%%%% Setting up the style

% \setlength\parindent{0pt} % Gets rid of all indentation
% \backgroundsetup{contents={\includegraphics[width=\textwidth]{ustc-name.pdf}},scale=0.4,placement=top,opacity=0.6,color=cyan,vshift=-20pt} %  USTC Logo

\pagestyle{fancy} % Enables the custom headers/footers

% 使用默认的Chapter页眉
% \lhead{} \rhead{} % Headers - all  empty

% \title{\vspace{-1.8cm}  \color{DarkRed} Laboratory Rotation Report}
% \subtitle{Title of the proposal % Title of the rotation project
% \vspace{-2cm} }
% \date{\today} % No date

\lfoot{\color{Grey} \textit{艾语晨}}  % Write your name here
\rfoot{ \color{Grey} 算法作业 }
\cfoot{\color{Grey} \thepage}

\renewcommand{\headrulewidth}{0.0pt} % No header rule
\renewcommand{\footrulewidth}{0.4pt} % Thin footer rule

\title{算法基础第六次作业}
\author{艾语晨~PB18000227}
\date{\today}

\linespread{1.3} %行间距为1.3倍默认间距 (1.3 x 1.2倍字符宽度)

\makeindex

\begin{document}
\theoremstyle{definition} \newtheorem{theorem}{Thm}[section] %定义一个定理Thm,序号为section的下一级序号
\theoremstyle{definition} \newtheorem{definition}{Def}[section] %定义一个定义Def,序号为section的下一级序号
\theoremstyle{plain} \newtheorem{lemma}{lemma}[section] %引理

	\maketitle
	\newpage

	\tableofcontents
	\newpage

	\chapter{}
	\section{聚合分析}
	\begin{quote}
		假定我们对一个数据结构执行一个由n个操作组成的操作序列,当i严格为2的幂时,第i个操作的代价为i,否则代价为1,请用聚合分析确定每个操作的摊还代价
	\end{quote}
	设$c_i$是第i个操作第代价,则由题意,有:
	\[c_i=\begin{cases}
		i\qquad i\mbox{是2的幂}\\
		1\qquad otherwise
	\end{cases}\]
	故n个操作的代价为:
	\[\sum_{i=1}^nc_i=\sum_{i=1}^{\lceil\lg n\rceil}+\sum_{i\le n\ \mbox{且不是2的幂}}1\le\sum_{i=1}^{\lceil\lg n\rceil}+n=2^{1+\lceil\lg n\rceil}-1+n\le 4n-1+n\le 5n\in O(n)\]
	由于要求平均,对以上结果除以n,得到时间复杂度为$O(1)$
	\section{核算法}
	给每一个操作赋以代价3。第一个操作有剩余的代价2。现在假设我们在执行完第$2^i$步之后还有非负的剩余代价,而接下来的$2^i-1$步都会剩余代价2,故加起来有$2\times(2^{i-1}-1)=2^{i+1}-1$的剩余代价。而在执行完第$2^{i+1}$步之后,剩余代价为$2^{i+1}-1+2-2^{i+1}=1$的剩余代价,依然非负。由数学归纳法可知,对任意步骤都有非负的代价。由于每一步的时间复杂度为$O(1)$,故整体的摊还复杂度为$O(n)$
	\section{势能法}
	当$i=2^k$时,令$\Phi(D_i)k+3$,否则,令k为满足$2^k\le i$的最大整数。则有$\Phi(D_i)=\Phi(D_{2^k})+2(i-2^k)$,同时定义$\Phi(D_0)=0$。那么,对于所有的$i\ge0$,$\Phi(D_i)\ge0$。势能差$\Phi(D_i)-\Phi(D_{i-1})$在$i=2^k$时是$-2^k+3$,否则为2,故n步操作的摊还代价为$\sum_{i=1}^n\hat{c}_i=3n=O(n)$
	\section{多维快速傅立叶变换}
	\subsection{a.}
	\[\begin{aligned}
		y_{k_1,\cdots,k_d}
		&=\sum_{j_1=0}^{n_1-1}\cdots\sum_{j_d=0}^{n_d-1}a_{j_1,\cdots,j_d}\omega_{n_1}^{j_1k_1}\cdots\omega_{n_d}^{j_dk_d}\\
		&=\sum_{j_d=0}^{n_d-1}\cdots\sum_{j_1=0}^{n_1-1}a_{j_1,\cdots,j_d}\omega_{n_1}^{j_1k_1}\cdots\omega_{n_d}^{j_dk_d}\\
		&=\sum_{j_d=0}^{n_d-1}\cdots\sum_{j_2=0}^{n_2-1}\left(\sum_{j_2=0}^{n_2-1}a_{j_1,\cdots,j_d}\omega_{n_1}^{j_1k_1}\right)\omega_{n_2}^{j_2k_2}\cdots\omega_{n_d}^{j_dk_d}\\
	\end{aligned}\]
	小括号里面的是一重傅立叶变换,而它需要对于每一个a的取值计算$n_2n_3\cdots n_d=n/n_1$次。将内层计算之后向外归纳,可知对于第k维,要做$n/\prod_{i<k}n_i$次DFT
	\subsection{b.}
	求和的顺序可以为任意,因为下标之间相互是独立的
	\subsection{c.}
	在第k维做每一次DFT的代价是$O(n_k\lg(n_k))$的,而它需要做总共$n/\prod_{i<k}n_i$次,故总时间代价为$O(n/\prod_{i<k}n_i\lg(n_k))$。
	\[\begin{aligned}
		\sum_{k=1}^dn/\left(\prod_{i<k}n_i\right)\lg(n_k)
		&\le \lg(n)\sum_{k=1}^dn/\left(\prod_{i<k}n_i\right)\\
		&\le \lg(n)\sum_{k=1}^dn/2^{k-1}\\
		&<n\lg n
	\end{aligned}\]与d独立

\end{document}
